\documentclass{article}

\usepackage[letterpaper, total={6in, 9in}]{geometry}
\usepackage{multicol}

\usepackage[tt=false, type1=true]{libertine}
% \usepackage[varqu]{zi4}

\usepackage{amsfonts}
\usepackage{amsmath}
\usepackage{amsthm}

\usepackage{titlesec}
\titleformat*{\section}{\Large\bfseries\sffamily}

\usepackage{graphicx}
\usepackage{booktabs}
\usepackage[colorlinks]{hyperref}
\usepackage{lipsum}

\usepackage{xcolor}

\usepackage{caption}
\usepackage{minted}
\setminted{
    autogobble=true,
    breaklines=true,
    fontsize=\small,
    xleftmargin=2em
}

\setminted[C]{
  frame=lines,
  framesep=2mm,
}

\setminted[Cpp]{
  frame=lines,
  framesep=2mm,
}

\setminted[bash]{
  frame=lines,
  framesep=2mm,
}

\renewcommand*{\subsectionautorefname}{Section}

\pagestyle{plain}


\title{\sffamily\bfseries
  CS5260 -- Spring 2022 \\
  Assignment 6
}

\author{
  Shen Jiamin \\
  A0209166A \\
  \href{mailto:shen_jiamin@u.nus.edu}{\nolinkurl{shen_jiamin@u.nus.edu}}
}

\date{}

\begin{document}
\maketitle

\begin{abstract}

In this assignment, we will study the use of LR range test, using
Colossalai framework. The chosen model and dataset are LeNet5 and
MNIST. You will be asked to:
\begin{enumerate}
    \item Choose one optimizer from SGD, ADAM, ADAMW, RADAM,
          LARS, LAMB or other optimizer you are interested in. (Note: AdaGrad
          is not supposed to work with this method).
    \item Choose two learning rate scheduling method from Pytorch library.
          (Including no scheduling) Some possible choice: Multistep, OneCycle
    \item Take use of \nolinkurl{Colossalai_lr_range_test.ipynb} to conduct LR range test
          for optimizer you chose. Propose several learning rates for real training.
    \item Write your code (either adapt provided code or start form scratch) to
          train LeNet5 on MNIST with two learning rate scheduling methods you
          choose and proposed learning rate. A suggested epoch number is 30.
    \item Observe the result, and write a brief report about what you find
          (optimizer and scheduling method you choose, corresponding LR region
          on LR range test plot. docx, pdf, ipynb are all acceptable). Remember to
          save necessary images or data during experiments for report writing.
    \item Upload your work including requirement.txt to your github. Add the
          github link to your report.
\end{enumerate}

\end{abstract}


% 另外一个问题是,这个notebook已经实现了lr range test,我们是改掉这一部分,用其他的scheduler重新实现这个过程吗

\end{document}
